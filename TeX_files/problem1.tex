\section{Problem no.1}

The task is to design a corporate beam forming network that provides the correct power distribution to each radiating element. The microstrip design is carried on considering FR4 as substrate, whose technological specifications follow:

 \begin{table} [h]
 	\label{tab:p1_FR4}
 	\centering	
 	\begin{tabular}{lrrr} 
 		\toprule 
 		Quantity & Symbol & value & \\
 		\midrule
		Dielectric permittivity &$\epsilon_{r}$		&4.3		& 		\\
 		Substrate height&	h				& 1.5		&  mm	\\ 
 		Metal thickness&t	& 0.1	& mm\\
 		Minimum strip width &W\textsubscript{min}&0.15 & mm \\
 		Minimum spacing between lines &$\Delta$s\textsubscript{min}&0.175 & mm \\
 		\bottomrule 
 	\end{tabular}	
 \end{table}

The antenna's radiating part requires four radiators with cosine-over-pedestal tapering, hence the BFN is symmetric with respect to the feeding line. %What follows is the theoretical design for the network, whose final optimized solution is presented in the following section. 

One possible implementation is proposed in FIGURE* PUT FIGURE WITH BFN'S BLOCK DIAGRAM, from which it appears that the design can be divided as follows: 
\begin{description}
	\item [Splitter1] this section matches the input line to Z\textsubscript{in}=50$\Omega$ at the input port. Moreover, it evenly splits the power into the two following branches, therefore the power splitter is a 3dB splitter. 
	\item [Splitter2] this section matches the radiating elements with section1 outputs, moreover it convey power into the radiators giving the correct power ratio between elements.
	\item [Splitter3] this is symmetric to section2 with respect to the feeding line longitudinal axis. 
	\item [Match1] these lines acts as matching connection from power splitter1 to power splitters2 and 3, moreover they are used to adjust the distance between elements.
	\item [Match2] matching connections between radiators and power splitters2,33;
\end{description}
Overall the BFN must be designed in order to be compliant with:
\begin{itemize}
	\item the requirement on grating lobes, then the element spacing d (d/$\lambda_0=0.724$);
	\item the requirement on amplitude tapering between edge and centre elements;
	\item the requirement on main beam direction. Since the array is broadside then the phase shift among elements must be null.
\end{itemize}
Each block's design and simulation are reported in the following paragraphs.

\subsection{Splitter 1 design}

Figure * PUT FIGURE OF SECTION 1 PHASE,IMP shows the block diagram of splitter 1. The input line is matched to the input port by means of a 50$\Omega$ line whose length has been chosen equal to $\lambda_{g}/4$. To evenly split the power into port 2 and 3 we use a T junction made up of two quarter-wavelength transformers. This components show real input and output impedance and the following relation for their characteristic impedance holds:
\begin{equation}
	\label{eq:p1_ZcQW}
	Z_{c,QW} =R_{c,QW}=\frac{1}{G_{c,QW}}= \sqrt{R_{in,QW}\cdot R_{out,QW}} 
\end{equation}
The two quarter-wavelength transformers are seen as two parallel admittances from the input port, then to have the 50$\Omega$ matching we have:
\begin{equation}
	G_{c,QW1}+G_{c,QW2}=\frac{1}{Z_{in}}=\frac{1}{50\Omega} \; \Rightarrow \; Z_{c,QW1}=Z_{c,QW2}=Z_{c,QW}=100\Omega \notag
\end{equation}
 Willing to reduce impedance discontinuities within the network, the section's output impedance is set to 50$\Omega$ too, hence:
\begin{equation}
	Z_{c,QW}= \sqrt{R_{in,QW}\cdot R_{out,QW}}=\sqrt{100\Omega\cdot 50\Omega}=70.7\Omega \notag
\end{equation}

The design has been converted from ideal microstrip lines to physical lines by means of TX-LINE: Transmission Line Calculator (from NI AWR Design Environment) and eventually implemented in CST DS\texttrademark{} for the optimization process. The theoretical and optimized dimensions are reported in table \ref{tab:p1_sec1DimIN} and\ref{tab:p1_sec1DimIN}. The resulting scattering parameters are shown in figure \ref{fig:p1_sec1Scatt}: both power splitting and phase behaves as expected. ADD DRAWING
\begin{table} [h]
	\label{tab:p1_sec1DimIN}
	\caption{Splitter1: Input line dimensions}
	\centering	
	\begin{tabular}{lccc} 
		\toprule
			& theoretical & optimized&\\
		\midrule 
		W\textsubscript{in} 	&	3.020		&	2.964	 & mm 		\\
		L\textsubscript{in}		&	16.797		& 	17.976		& mm	\\ 
		Z\textsubscript{c,in}	& 	50 & 50.6	&$\Omega$ \\
		\bottomrule
	\end{tabular}	
\end{table}

\begin{table} [h]
	\label{tab:p1_sec1DimQW}
	\caption{Splitter1: quarter-wavelength transformer dimensions}
	\centering	
	\begin{tabular}{lccc} 
		\toprule
		& theoretical & optimized&\\
		\midrule 
		W\textsubscript{QW} 	&	1.599		&	1.458	& mm		\\
		L\textsubscript{QW}	&	17.248		& 	16.706		& mm	\\ 
		Z\textsubscript{c,QW}& 	70.7 &73.9	&$\Omega$ \\
		\bottomrule
	\end{tabular}	
\end{table}

\begin{figure}[h] 
	\centering
	\subfloat[][\emph{Modulus}]{\includegraphics[scale=.6,angle=90]{p1_sec1_Smod}}\quad
	\subfloat[][\emph{Phase}]{\includegraphics[scale=.6,angle=90]{p1_sec1_Sph}}\\
	\caption{Section 1 most important scattering parameters. As it appears both input matching and even power splitting are accomplished, the phase shift is null (reference impedance 50$\Omega$).}
	\label{fig:p1_sec1Scatt}
\end{figure}

\newpage

\subsection{Splitters2,3 and match2 design}

Figure * PUT FIGURE OF SECTION 2 PHASE,IMP shows the block diagram of splitter2 and match1 (splitter3  and match1 are specular).

Match1 lines are designed as $\lambda_{g}/2$, 50$\Omega$ characteristic impedance microstrip lines. Acting as transparent transmission lines, they allow splitter2 and 3 to have Z\textsubscript{in}=50$\Omega$ at their input ports. Again a Z\textsubscript{out}=50$\Omega$ is chosen to avoid abrupt impedance variations. Imposing Z\textsubscript{in}=50$\Omega$ we also avoid the appearance of lines narrower than W\textsubscript{min} within the power splitter2 (and splitter3).

The splitter is composed by two quarter-wavelength transformers that provide the correct current tapering for the radiators. To design that we will consider the power entering and exiting the component.
Having chosen as tapering t=0.54, then:
\begin{equation}
	\frac{S_{31}}{S_{21}}=0.54 \; \Rightarrow \; \frac{P_3}{P_2}=\Bigg(\frac{S_{31}}{S_{21}}\Bigg)^2=0.292 \Rightarrow \frac{P_3}{P_2}=-5.34dB \notag
\end{equation} 
where S\textsubscript{21} and S\textsubscript{31} are the scattering transmission coefficients at port 2 and 3. In particular P\textsubscript{2} is the power exiting the splitter and directed to the centre element, whereas P\textsubscript{3} is goes to the edge radiator (supposing lossless TXL):
\begin{equation}
	\begin{cases}
		P_2 = P_a = \frac{1}{2}R_{in,QWa}|I_A|^2 \notag \\	
		P_3 = P_b = \frac{1}{2}R_{in,QWb}|I_B|^2 \notag	
	\end{cases}	
\end{equation}
As before, the two impedance transformers are seen as two in-parallel conductances from port 1, then the following system holds: 
\begin{equation}
	\begin{cases} 
		\frac{1}{R_{in,QWa}}+\frac{1}{R_{in,QWb}}=\frac{1}{Z_{in}} \notag\\
		\frac{P_{3}}{P_{2}}=\frac{R_{in,QWa}}{R_{in,QWa}}=0.292 \notag
	\end{cases}
\end{equation}
that has solutions:
\begin{gather}
	R_{in,QWa} = 221\Omega \notag \\
	R_{in,QWb} = 64.6\Omega \notag
\end{gather}
From equation (\ref{eq:p1_ZcQW}):
\begin{gather}
	Z_{c,QWa}=\sqrt{R_{in,QWa}\cdot Z_{out}}=\sqrt{221\Omega \cdot 50\Omega}=105\Omega \notag \\
	Z_{c,QWb}=\sqrt{R_{in,QWb}\cdot Z_{out}}=\sqrt{64.4\Omega \cdot 50\Omega}=56.7\Omega \notag
\end{gather}
The circuit has been then converted into microstrip layout and simulated. The Optimized dimensions are reported in table \ref{tab:p2_sec2DimM1}, \ref{tab:p2_sec2DimQWa} and \ref{tab:p2_sec2DimQWb}. Unfortunately, the splitter2's values found so far produced a wrong tapering (t=-4.2dB), therefore the lines width and length have been tuned till the desired tapering specification has been fulfilled. Since t>-5.3dB we need to increase the difference between R\textsubscript{in,QWa} and R\textsubscript{in,QWb}. 
With t=-5.2dB we have the best compromise between excitation amplitude distribution and phase shift, that is equal to 3.94$^\circ$.
 Figure \ref{fig:p2_sec1Scatt} shows the input parameters. ADD DRAWING

\begin{table} [h]
	\label{tab:21_sec2DimM1}
	\caption{Match1: microstrip dimensions.}
	\centering	
	\begin{tabular}{lcccc} 
		\toprule
		& theoretical & optimized (no tap)& optimized (tap)&\\
		\midrule 
		W\textsubscript{M1} 	&	3.021		&	2.964	&2.695& mm 		\\
		L\textsubscript{M1}		&	33.594		& 	33.594	&23.133& mm		\\ 
		Z\textsubscript{c,M1}	&	50			& 	50.6	&53.5& $\Omega$		\\
		\bottomrule
	\end{tabular}	
\end{table}
\begin{table} [h!]
	\label{tab:p2_sec2DimQWa}
	\caption{Splitter2: quarter-wavelength a dimensions.}
	\centering	
	\begin{tabular}{lcccc} 
		\toprule
		& theoretical & optimized(tap) &optimized (tap)&\\
		\midrule 
		W\textsubscript{QWa} 	&	2.435		&	2.106	&2.295& mm		\\
		L\textsubscript{QWa}	&	16.959		& 	14.656	&15.262& mm		\\ 
		Z\textsubscript{c,QWa}	&	56.6		& 	61.4	&58.6& $\Omega$		\\
		\bottomrule
	\end{tabular}	
\end{table}
\begin{table} [h!]
	\label{tab:p2sec2DimQWb}
	\caption{Splitter2: quarter-wavelength b dimensions.}
	\centering	
	\begin{tabular}{lcccc} 
		\toprule
		& theoretical & optimized(no tap) &optimized (tap)&\\
		\midrule 
		W\textsubscript{QWb} 	&	0.613		&	0.507	&0.457& mm		\\
		L\textsubscript{QWb}	&	17.709		& 	16.170	&14.005& mm		\\ 
		Z\textsubscript{c,QWb}	&	105			& 	112		&116& $\Omega$		\\
		\bottomrule
	\end{tabular}	
\end{table}
\newpage

\begin{figure}[h!] 
	\centering
	\subfloat[][\emph{Modulus}]{\includegraphics[scale=.6,angle=90]{p1_sec2_Smod}}\quad
	\subfloat[][\emph{Phase}]{\includegraphics[scale=.6,angle=90]{p1_sec2_Sph}}\\
	\caption{Section 1 most important scattering parameters. As it appears both input matching and even power splitting are accomplished. The tapering value is t=-5.2dB, the phase shift amount to 3.94$^\circ$ (reference impedance 50$\Omega$).}.
	\label{fig:p2_sec1Scatt}
\end{figure}

\subsection{Match2 design}

As final design step another quarter-wavelength transformer is introduced. This element matches splitter2 and splitter3, R\textsubscript{out}=50$\Omega$, output impedance with the radiators' input impedance, equal to Z\textsubscript{in,patch}=120$\Omega$. Using equation (\ref{eq:p1_ZcQW}) we have:
\begin{align}
	Z_{c,M2}&=\sqrt{R_{in,M2}\cdot R_{out,M2}} \notag\\
	&=\sqrt{Z_{out}\cdot Z_{in,patch}} \notag\\
	&=\sqrt{120\Omega \cdot 50\Omega}=77.5\Omega \notag
\end{align}
The conversion from ideal microstrip to layout can be found in table \ref{tab:21_DimM2}.

\begin{table} [h]
	\label{tab:21_DimM2}
	\caption{Match2: microstrip dimensions.}
	\centering	
	\begin{tabular}{lccc} 
		\toprule
		& theoretical 			& optimized &\\
		\midrule 
		W\textsubscript{M1} 	&	1.316		&	1.212	& mm 		\\
		L\textsubscript{M1}		&	17.365		& 	17.365	& mm		\\ 
		Z\textsubscript{c,M1}	&	77.5		& 	80.45	& $\Omega$		\\
		\bottomrule
	\end{tabular}	
\end{table}

\begin{figure}[t] 
	\centering
	\includegraphics[scale=0.5]{p1_M2_Smod}
	\caption{Match2 return loss (reference impedance 50$\Omega$). }
	\label{fig:e1_array}
\end{figure}