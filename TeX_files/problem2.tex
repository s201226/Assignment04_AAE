\section{Problem no.2}

In order to verify and optimize the design of the BFN we decided to use \textit{CST Design Studio}, one of the various tools offered by \textit{CST Studio Suite}, which allows the user to design a microstrip network via schematic, to optimize it through the embedded optimization tool and then to create a layout of the whole project that can easily be exported in \textit{CST Microwave Studio} for further simulations.

\par\medskip
\noindent
Figure \ref{BFN_schematic} shows how a schematic looks like in \textit{CST Design Studio}. The isolated block is needed in order to assign to the structure fundamental properties such as thickness of the ground plane and dielectric constant of the substrate, while the other blocks build up the whole structure, allowing the creation of lines with assigned width and length and even bends (for example mitered) and junctions.

\begin{figure}[H]
\centering
\includegraphics[scale=0.4]{BFN_schematic.png}
\caption{BFN schematic on \textit{CST Design Studio}}
\label{BFN_schematic}
\end{figure}

\par\medskip
\noindent
As a first step the BFN has been drawn with the dimensions calculated during the design phase and then we specified that we wanted to see the S parameters at all ports (port 1 has an impedance of 50 $\Omega$, ports 2 to 5 have an impedance of 120 $\Omega$, since they take the place of the patches) as a simulation result. Next, the optimizer was set with the initial goal of tuning the S\textsubscript{11} parameter so to make it resonate at 2.45 GHz with the highest possible precision, obtaining the result shown in figure \ref{BFN_S11}.

\begin{figure}[H]
\centering
\includegraphics[scale=0.35]{BFN_S11.png}
\caption{S\textsubscript{11} parameter of our BFN}
\label{BFN_S11}
\end{figure}

\par\medskip
\noindent
As one can see, S\textsubscript{11} is pretty well shaped and furthermore the resonance at the desired frequency is strong and the obtained bandwidth is quite large.

\par\medskip
\noindent
Next step is the verification of the obtained tapering and phasing, which can be done by checking the value of the S\textsubscript{ij} parameters.

\begin{figure}[H]
\centering
\includegraphics[scale=0.35]{S21Amp.png}
\caption{S\textsubscript{21} parameter of our BFN}
\label{S21Amp}
\end{figure}

\begin{figure}[H]
\centering
\includegraphics[scale=0.35]{S31Amp.png}
\caption{S\textsubscript{31} parameter of our BFN}
\label{S31Amp}
\end{figure}

\begin{figure}[H]
\centering
\includegraphics[scale=0.35]{S41Amp.png}
\caption{S\textsubscript{41} parameter of our BFN}
\label{S41Amp}
\end{figure}

\begin{figure}[H]
\centering
\includegraphics[scale=0.35]{S51Amp.png}
\caption{S\textsubscript{51} parameter of our BFN}
\label{S51Amp}
\end{figure}

\par\medskip
\noindent
The power ratio between the central elements of the array (ports 3 and 4) and the edge ones (ports 1 and 5) turns out to be 5.98 dB.

\par\medskip
\noindent
Last but not least, we have to verify that the phase shift between the output ports is as close as possible to 0.

\begin{figure}[H]
\centering
\includegraphics[scale=0.35]{S21p.png}
\caption{S\textsubscript{21} parameter (phase), coinciding with S\textsubscript{51}}
\label{S21phase}
\end{figure}

\begin{figure}[H]
\centering
\includegraphics[scale=0.35]{S31p.png}
\caption{S\textsubscript{31} parameter (phase), coinciding with S\textsubscript{41}}
\label{S31phase}
\end{figure}

\par\medskip
\noindent
The phase difference between center and edge elements is around 4\textsuperscript{$\circ$}. A simulation of the complete structure in \textit{Microwave Studio} will tell us whether this lack of precision will affect the behavior of the array a lot or it will result either way to be broadside, as expected.

\par\medskip
\noindent
In order to have a visual impact of what the schematic is translated into in terms of layout, figure \ref{layout} shows the resulting structure.

\begin{figure}[H]
\centering
\includegraphics[scale=0.45]{layout.png}
\caption{How our BFN looks like}
\label{layout}
\end{figure}